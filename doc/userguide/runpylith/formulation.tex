\section{Time-Dependent Problem (\facilityshape{formulation})}

This type of problem applies to transient static, quasi-static, and
dynamic simulations. The time-dependent problem adds the
\facility{formulation} facility to the general-problem. The
formulation specifies the time-stepping formulation to integrate the
elasticity equation. PyLith provides several alternative formulations,
each specific to a different type of problem.
\begin{description}
\item[\object{Implicit}] Implicit time stepping for static and
   quasi-static problems with infinitesimal strains. The implicit
   formulation neglects inertial terms (see Section
   \vref{eq:elasticity:integral:quasistatic}).
\item[\object{ImplicitLgDeform}] Implicit time stepping for static
  and quasi-static problems including the effects of rigid body motion
  and small strains.  This formulation requires the use of the
  nonlinear solver, which is selected automatically.
\item[\object{Explicit}] Explicit time stepping for dynamic problems
  with infinitesimal strains and lumped system Jacobian. The cell
  matrices are lumped before assembly, permitting use of a vector for
  the diagonal system Jacobian matrix. The built-in lumped solver is
  selected automatically.
\item[\object{ExplicitLgDeform}] Explicit time stepping for dynamic
  problems including the effects of rigid body motion and small
  strains. The cell matrices are lumped before assembly, permitting
  use of a vector for the diagonal system Jacobian matrix. The
  built-in lumped solver is selected automatically.
\item[\object{ExplicitTri3}] Optimized elasticity formulation for
  linear triangular cells with one point quadrature for dynamic
  problems with infinitesimal strains and lumped system Jacobian. The
  built-in lumped solver is selected automatically.
\item[\object{ExplicitTet4}] Optimized elasticity formulation for
  linear tetrahedral cells with one point quadrature for dynamic
  problems with infinitesimal strains and lumped system Jacobian. The
  built-in lumped solver is selected automatically.
\end{description}
In many quasi-static simulations it is convenient to compute a static
problem with elastic deformation prior to computing a transient response.
Up through PyLith version 1.6 this was hardwired into the Implicit
Forumulation as advancing from time step $t=-\Delta t$ to $t=0$,
and it could not be turned off. PyLith now includes a property, \property{elastic\_prestep}
in the TimeDependent component to turn on/off this behavior (the default
is to retain the previous behavior of computing the elastic deformation).

\warning{Turning off the elastic
prestep calculation means the model only deforms when an {\it increment}
in loading or deformation is applied, because the time-stepping formulation
is implemented using the increment in displacement.}

The \object{TimeDependent} properties and facilities include
\begin{inventory}
  \propertyitem{elastic\_preset}{If true, perform a static calculation with elastic
    behavior before time stepping (default is True).}
  \facilityitem{formulation}{Formulation for solving the partial differential
    equation.}
\end{inventory}

\begin{cfg}[\object{TimeDependent} parameters in a \filename{cfg} file]
<h>[pylithapp.timedependent]</h>
<f>formulation</f> = pylith.problems.Implicit ; default
<f>progres_monitor</f> = pylith.problems.ProgressMonitorTime ; default
<p>elastic_preset</p> = True ; default
\end{cfg}
The formulation value can be set to the other formulations in a similar
fashion. 


\subsection{Time-Stepping Formulation}

The explicit and implicit time stepping formulations use a common
set of facilities and properties. The properties and facilities include
\begin{inventory}
\propertyitem{matrix\_type}{Type of PETSc matrix for the system Jacobian (sparse
matrix, default is symmetric, block matrix with a block size of 1).}
\propertyitem{view\_jacobian}{Flag to indicate if system Jacobian (sparse matrix)
should be written to a file (default is false).}
\propertyitem{split\_fields}{Split solution field into a displacement portion
(fields 0..ndim-1) and a Lagrange multiplier portion (field ndim)
to permit application of sophisticated PETSc preconditioners (default
is false).}
\facilityitem{time\_step}{Time step size specification (default is \object{TimeStepUniform} (uniform time step).}
\facilityitem{solver}{Type of solver to use (default is \object{SolverLinear}).}
\facilityitem{output}{Array of output managers for output of the solution (default
is [output]).}
\facilityitem{jacobian\_viewer}{Viewer to dump the system Jacobian (sparse matrix)
to a file for analysis (default is PETSc binary).}
\end{inventory}

\begin{cfg}[Time-stepping formulation parameters in a \filename{cfg} file]
<h>[pylithapp.timedependent.formulation]</h>
<p>matrix_type</p> = sbaij ; Non-symmetric sparse matrix is 'aij'
<p>view_jacobian</p> = false

# Nonlinear solver is pylith.problems.SolverNonlinear
<f>solver</f> = pylith.problems.SolverLinear
<f>output</f> = [domain, ground_surface]
<f>time_step</f> = pylith.problems.TimeStepUniform
\end{cfg}

\subsection{Numerical Damping in Explicit Time Stepping}

In explicit time-stepping formulations for elasticity, boundary conditions
and fault slip can excite short waveform elastic waves that are not
accurately resolved by the discretization. We use numerical damping
via an artificial viscosity\cite{Knopoff:Ni:2001,Day:Ely:2002} to
reduce these high frequency oscillations. In computing the strains
for the elasticity term in equation \vref{eq:elasticity:integral:dynamic:t},
we use an adjusted displacement rather than the actual displacement,
where 
\begin{equation}
\vec{u}^{adj}(t)=\vec{u}(t)+\eta^{*}\Delta t\vec{\dot{u}}(t),
\end{equation}
$\vec{u}^{adj}(t)$ is the adjusted displacement at time t, $\vec{u}(t)$is
the original displacement at time (t), $\eta^{*}$is the normalized
artificial viscosity, $\Delta t$ is the time step, and $\vec{\dot{u}}(t)$
is the velocity at time $t$. The default value for the normalized
artificial viscosity is 0.1. We have found values in the range 0.1-0.4
sufficiently suppress numerical noise while not excessively reducing
the peak velocity. An example of setting the normalized artificial
viscosity in a \filename{cfg} file is
\begin{cfg}
<h>[pylithapp.timedependent.formulation]</h>
<p>norm_viscosity</p> = 0.2
\end{cfg}

\subsection{Solvers}
\label{sec:solvers}

PyLith supports three types of solvers. The linear solver,
SolverLinear, corresponds to the PETSc KSP solver and is used in
linear problems with linear elastic and viscoelastic bulk constitutive
models and kinematic fault ruptures. The nonlinear solver,
SolverNonlinear, corresponds to the PETSc SNES solver and is used in
nonlinear problems with nonlinear viscoelastic or elastoplastic bulk
constitutive models, dynamic fault ruptures, or problems involving
finite strain (small strain formulation).  The lumped solver
(SolverLumped) is a specialized solver used with the lumped system
Jacobian matrix. The options for the PETSc KSP and SNES solvers are
set via the top-level PETSc options (see Section
\vref{sec:petsc:options} and the PETSc documentation
\url{www.mcs.anl.gov/petsc/petsc-as/documentation/index.html}).


\subsection{Time Stepping}
\label{sec:time-stepping}

PyLith provides three choices for controlling the time step in time-dependent
simulations. These include (1) a uniform, user-specified time step
(which is the default), (2) user-specified time steps (potentially
nonuniform), and (3) automatically calculated (potentially nonuniform)
time steps. The procedure for automatically selecting time steps requires
that the material models provide a reasonable estimate of the time
step for stable time integration. In general, quasi-static simulations
with viscoelastic materials should use automatically calculated time
steps and dynamic simulations should use a uniform, user-specified
time step. Note that all three of the time stepping schemes make use
of the computed stable time step (see \vref{sec:stable:time:step}).
When using user-specified time steps, the value is checked against
the computed stable time step. The automatically calculated time step
comes from the computed stable time step.

\warning{Varying the time step within a simulation requires
  recomputing the Jacobian of the system whenever the time step
  changes, which can greatly increase the runtime if the time-step
  size changes frequently.}

\subsubsection{Uniform, User-Specified Time Step (\object{TimeStepUniform})}

With a uniform, user-specified time step, the user selects the time
step that is used over the entire duration of the simulation. If this
value exceeds the computed stable time step at any time, PyLith will
terminate with an error. The properties for the uniform, user-specified
time step are:
\begin{inventory}
\propertyitem{total\_time}{Time duration for simulation (default is 0.0 s).}
\propertyitem{start\_time}{Start time for simulation (default is 0.0 s).}
\propertyitem{dt}{Time step for simulation.}
\end{inventory}

\begin{cfg}[\object{TimeStepUniform} parameters in a \filename{cfg} file]
<h>[pylithapp.problem.formulation]</h>
<p>time_step</p> = pylith.problems.TimeStepUniform ; Default value

<h>[pylithapp.problem.formulation.time_step]</h>
<p>total_time</p> = 1000.0*year
<p>dt</p> = 0.5*year
\end{cfg}

\subsubsection{Nonuniform, User-Specified Time Step (\object{TimeStepUser})}

The nonuniform, user-specified, time-step implementation allows the
user to specify the time steps in an ASCII file (see Section
\vref{sec:format:TimeStepUser} for the format specification of the
time-step file). If the total duration exceeds the time associated
with the time steps, then a flag determines whether to cycle through
the time steps or to use the last specified time step for the time
remaining. Similar to the uniform time step, if the user-specified
time step size exceeds the computed stable time step at any time,
PyLith will terminate with an error.  The properties for the
nonuniform, user-specified time step are:
\begin{inventory}
\propertyitem{total\_time}{Time duration for simulation.}
\propertyitem{filename}{Name of file with time-step sizes.}
\propertyitem{loop\_steps}{If true, cycle through time steps, otherwise keep
using last time-step size for any time remaining.}
\end{inventory}

\begin{cfg}[\object{TimeStepUser} parameters in a \filename{cfg} file]
<h>[pylithapp.problem.formulation]</h>
<f>time_step</f> = pylith.problems.TimeStepUser ; Change the time step algorithm

<h>[pylithapp.problem.formulation.time_step]</h>
<p>total_time</p> = 1000.0*year
<p>filename</p> = timesteps.txt
<p>loop_steps</p> = false ; Default value
\end{cfg}

\subsubsection{Nonuniform, Automatic Time Step (\object{TimeStepAdapt})}

This time-step implementation automatically calculates a time step
size based on the constitutive model and rate of deformation. As a
result, this choice for choosing the time step relies on accurate
calculation of a stable time step within each finite-element cell
by the constitutive models. To provide some control over the time-step
selection, the user can control the frequency with which a new time
step is calculated, the time step to use relative to the value determined
by the constitutive models, and a maximum value for the time step.
Note that the stability factor allows the computed time step size
to exceed the computed stable time step. A stability factor of 1.0
would provide a time step size equal to the stable time step, while
a value of 2.0 (default value) would provide a time step size equal
to 1/2 the stable time step. Caution should be used when adjusting
the stability factor to values less than 1.0, as the large time step
size may result in inaccurate solutions. The properties for controlling
the automatic time-step selection are:
\begin{inventory}
\propertyitem{total\_time}{Time duration for simulation.}
\propertyitem{max\_dt}{Maximum time step permitted.}
\propertyitem{adapt\_skip}{Number of time steps to skip between calculating
new stable time step.}
\propertyitem{stability\_factor}{Safety factor for stable time step (default
is 2.0).}
\end{inventory}

\begin{cfg}[\object{TimeStepAdapt} parameters in a \filename{cfg} file]
<h>[pylithapp.problem.formulation]</h>
<p>time_step</p> = pylith.problems.TimeStepAdapt ; Change the time step algorithm

<h>[pylithapp.problem.formulation.time_step]</h>
<p>total_time</p> = 1000.0*year
<p>max_dt</p> = 10.0*year
<p>adapt_skip</p> = 10 ; Default value
<p>stability_factor</p> = 2.0 ; Default value
\end{cfg}

\section{Green's Functions Problem (\object{GreensFns})}

This type of problem applies to computing static Green's functions
for elastic deformation. The \object{GreensFns} problem specializes
the time-dependent facility to the case of static simulations with
slip impulses on a fault. The default formulation is the Implicit
formulation and should not be changed as the other formulations are
not applicable to static Green's functions. In the output files, the
deformation at each ``time step'' is the deformation for a different
slip impulse. The properties provide the ability to select which fault
to use for slip impulses. The only fault component available for use
with the \object{GreensFns} problem is the \object{FaultCohesiveImpulses}
component discussed in Section \vref{sec:fault:cohesive:impulses}.
The \object{GreensFns} properties amd facilities include:
\begin{inventory}
\propertyitem{fault\_id}{Id of fault on which to impose slip impulses.}
\propertyitem{formulation}{Formulation for solving the partial differential
equation.}
\propertyitem{progress\_monitor}{Simple progress monitor via text file.}
\end{inventory}

\begin{cfg}[\object{GreensFns} parameters in a \filename{cfg} file]
<h>[pylithapp]</h>
<f>problem</f> = pylith.problems.GreensFns ; Change problem type from the default

<h>[pylithapp.greensfns]</h>
<p>fault_id</p> = 100 ; Default value
<f>formulation</f> = pylith.problems.Implicit ; default
<f>progres_monitor</f> = pylith.problems.ProgressMonitorTime ; default
\end{cfg}

\warning{The \object{GreensFns} problem generates slip impulses on a
  fault. The current version of PyLith requires that impulses can only
  be applied to a single fault and the fault facility must be set to
  \object{FaultCohesiveImpulses}.}

\section{Progress Monitors}
\newfeature{v2.1.0}

The progress monitors make it easy to monitor the general progress of
long simulations, especially on clusters where stdout is not always
easily accessible. The progress monitors update a simulation's current
progress by writing information to a text file. The information
includes time stamps, percent completed, and an estimate of when the
simulation will finish.

\subsection{\object{ProgressMonitorTime}}

This is the default progress monitor for time-stepping problems. The
monitor calculates the percent completed based on the time at the
current time step and the total simulated time of the simulation,
not the total number of time steps (which may be unknown in simulations
with adaptive time stepping). The \object{ProgressMonitorTime} properties
include:
\begin{inventory}
\propertyitem{update\_percent}{Frequency (in percent) of progress updates.}
\propertyitem{filename}{Name of output file.}
\propertyitem{t\_units}{Units for simulation time in output.}
\end{inventory}

\begin{cfg}[\object{ProgressMonitorTime} parameters in a \filename{cfg} file]
<h>[pylithapp.problem.progressmonitor]</h>
<p>update_percent</p> = 5.0 ; default
<p>filename</p> = progress.txt ; default
<p>t_units</p> = year ; default
\end{cfg}

\subsection{\object{ProgressMonitorStep}}

This is the default progress monitor for problems with a specified
number of steps, such as Green's function problems. The monitor calculates
the percent completed based on the number of steps (e.g., Green's
function impulses completed). The ProgressMonitorStep propertiles
include:
\begin{inventory}
\propertyitem{update\_percent}{Frequency (in percent) of progress updates.}
\propertyitem{filename}{Name of output file.}
\end{inventory}

\begin{cfg}[\object{ProgressMonitorStep} parameters in a \filename{cfg} file]
<h>[pylithapp.problem.progressmonitor]</h>
<p>update_percent</p> = 5.0 ; default
<p>filename</p> = progress.txt ; default
\end{cfg}

% End of file
